Para conseguir completar este proyecto se ha hecho un plan de trabajo con el que se pretende tener un control sobre el tiempo empleado en cada fase del desarrollo del proyecto, poniendo fechas límite que ayuden a completar los objetivos a tiempo.

\begin{itemize}
    \item \textbf{Del 1 de febrero al 1 de marzo:} se fijan los objetivos del proyecto y se estudia el problema.
    \item \textbf{Del 1 de marzo al 15 de marzo:} se define la estructura del proyecto y de la memoria. Se decide qué se explicará en la memoria y cómo se pretende presentar esta información.
    \item \textbf{Del 15 de marzo al 15 de junio:} se desarrolla el proyecto. En esta fase se empezará el trabajo efectivo del proyecto, se construirá la placa de control y el brazo robótico, se desarrollará el \textit{software} de control y se integrarán los distintos subsistemas. La memoria se desarrollará a la par que la parte práctica del proyecto.
    \item \textbf{Día 15 de junio:} se realiza la primera entrega oficial al tutor de la asignatura.
    \item \textbf{Del 15 de junio a 1 de julio:} periodo de revisión y mejora del proyecto. Se espera retroalimentación del tutor y se mejorará el proyecto en base a dicha retroalimentación.
    \item \textbf{Día 1 de julio:} convocatoria del TFG.
    \item \textbf{Del 1 de julio al 15 de julio:} revisión por parte del tribunal, se realizan los tramites administrativos necesarios y se imprimen los ejemplares necesarios del TFG.
    \item \textbf{Día 15 de julio:} lectura del TFG por parte del equipo de desarrollo ante el tribunal.
    \item \textbf{Día 16 de julio:} final del proyecto.
\end{itemize}

Las fechas y los periodos anteriormente mencionados están sujetos a cambios a lo largo del desarrollo del proyecto para poder adaptarnos a imprevistos que puedan surgir o a limitaciones ajenas al equipo de trabajo.