%%%%%%%%%%%%%%%%%%%%%%%%%%%%%%%%%%%%%%%%%%%%%%%%%%%%%%%%%%%%%%%%%%%%%%%%%%%%%%%%%

%%%%%%%%%%%%%%%%%%%%%%%%%%%%%%%%%%%%%%%%%%%%%%%%%%%%%%%%%%%%%%%%%%%%%%%%%%%%%%%%%
%% Documenclass 
%%%%%%%%%%%%%%%%%%%%%%%%%%%%%%%%%%%%%%%%%%%%%%%%%%%%%%%%%%%%%%%%%%%%%%%%%%%%%%%%%
\documentclass[a4paper,oneside,titlepage,12pt]{article}
%%%%%%%%%%%%%%%%%%%%%%%%%%%%%%%%%%%%%%%%%%%%%%%%%%%%%%%%%%%%%%%%%%%%%%%%%%%%%%%%%
%% Packages
%%%%%%%%%%%%%%%%%%%%%%%%%%%%%%%%%%%%%%%%%%%%%%%%%%%%%%%%%%%%%%%%%%%%%%%%%%%%%%%%%
\usepackage[spanish]{babel}
\usepackage{multirow}
\usepackage{amsmath}
\usepackage{complexity}
\usepackage[T1]{fontenc}
\usepackage[utf8]{inputenc}
\usepackage[pdftex]{graphicx} %%Graphics in pdfLaTeX
\usepackage{csquotes}
\usepackage{a4wide} %%Smaller margins, more text per page.
\usepackage{longtable} %%For tables that exceed a page width
\usepackage{pdflscape} %%Adds PDF sup­port to the land­scape en­vi­ron­ment of pack­age
\usepackage{caption} %%Pro­vides many ways to cus­tomise the cap­tions in float­ing en­vi­ron­ments like fig­ure and ta­ble
\usepackage{float} %%Im­proves the in­ter­face for defin­ing float­ing ob­jects such as fig­ures and ta­bles
\usepackage[tablegrid,nochapter]{vhistory} %%Vhis­tory sim­pli­fies the cre­ation of a his­tory of ver­sions of a doc­u­ment
\usepackage[nottoc]{tocbibind} %%Au­to­mat­i­cally adds the bib­li­og­ra­phy and/or the in­dex and/or the con­tents, etc., to the Ta­ble of Con­tents list­ing
\usepackage[toc,page]{appendix} %%The ap­pendix pack­age pro­vides var­i­ous ways of for­mat­ting the ti­tles of ap­pen­dices
\usepackage{pdfpages} %%This pack­age sim­pli­fies the in­clu­sion of ex­ter­nal multi-page PDF doc­u­ments in LATEX doc­u­ments
\usepackage[rightcaption]{sidecap} %%De­fines en­vi­ron­ments called SC­fig­ure and SCtable (anal­o­gous to fig­ure and ta­ble) to type­set cap­tions side­ways
% \usepackage{cite} %%The pack­age sup­ports com­pressed, sorted lists of nu­mer­i­cal ci­ta­tions, and also deals with var­i­ous punc­tu­a­tion and other is­sues of rep­re­sen­ta­tion, in­clud­ing com­pre­hen­sive man­age­ment of break points
\usepackage[printonlyused]{acronym} %%This pack­age en­sures that all acronyms used in the text are spelled out in full at least once. It also pro­vides an en­vi­ron­ment to build a list of acronyms used
\usepackage[pdftex,scale={.8,.8}]{geometry} %%The pack­age pro­vides an easy and flex­i­ble user in­ter­face to cus­tomize page lay­out, im­ple­ment­ing auto-cen­ter­ing and auto-bal­anc­ing mech­a­nisms so that the users have only to give the least de­scrip­tion for the page lay­out. For ex­am­ple, if you want to set each mar­gin 2cm with­out header space, what you need is just \usep­a­ck­age[mar­gin=2cm,no­head]{ge­om­e­try}.
\usepackage{layout} %%The pack­age de­fines a com­mand \lay­out, which will show a sum­mary of the lay­out of the cur­rent doc­u­ment
\usepackage{subfigure} %%Pro­vides sup­port for the ma­nip­u­la­tion and ref­er­ence of small or ‘sub’ fig­ures and ta­bles within a sin­gle fig­ure or ta­ble en­vi­ron­ment.
\usepackage[toc]{glossaries} %%The glos­saries pack­age sup­ports acronyms and mul­ti­ple glos­saries, and has pro­vi­sion for op­er­a­tion in sev­eral lan­guages (us­ing the fa­cil­i­ties of ei­ther ba­bel or poly­glos­sia).
\usepackage[left,pagewise,modulo]{lineno} %%Adds line num­bers to se­lected para­graphs with ref­er­ence pos­si­ble through the LATEX \ref and \pageref cross ref­er­ence mech­a­nism
\usepackage[pdftex,colorlinks=false,hidelinks,pdfstartview=FitV,bookmarks=true]{hyperref}%%The hy­per­ref pack­age is used to han­dle cross-ref­er­enc­ing com­mands in LATEX to pro­duce hy­per­text links in the doc­u­ment. 
\usepackage{metainfo}
\usepackage[pagestyles,raggedright]{titlesec}
\usepackage{etoolbox}
\usepackage{tabularx}
\usepackage{ltxtable}
\usepackage[style=ieee]{biblatex}
\bibliography{base/sources.bib}
\usepackage[official]{eurosym}
\usepackage{gensymb}
\usepackage{%
	array, %%An ex­tended im­ple­men­ta­tion of the ar­ray and tab­u­lar en­vi­ron­ments which ex­tends the op­tions for col­umn for­mats, and pro­vides "pro­grammable" for­mat spec­i­fi­ca­tions
	booktabs, %%The pack­age en­hances the qual­ity of ta­bles in LATEX, pro­vid­ing ex­tra com­mands as well as be­hind-the-scenes op­ti­mi­sa­tion
	dcolumn, %%
	rotating,
	shortvrb,
	units,
	url,
	lastpage,
	longtable,
	lscape,
	qtree,
	skmath,	
	enumitem,
}
\usepackage{titling}
\newcommand{\subtitle}[1]{%
  \posttitle{%
    \par\end{center}
    \begin{center}\large#1\end{center}
    \vskip0.5em}%
}
%%%%%%%%%%%%%%%%%%%%%%%%%%%%%%%%%%%%%%%%%%%%%%%%%%%%%%%%%%%%%%%%%%%%%%%%%%%%%%%%%
%% Java --> latex 
%%%%%%%%%%%%%%%%%%%%%%%%%%%%%%%%%%%%%%%%%%%%%%%%%%%%%%%%%%%%%%%%%%%%%%%%%%%%%%%%%
\usepackage{listings}
\usepackage{color}
\definecolor{pblue}{rgb}{0.13,0.13,1}
\definecolor{pgreen}{rgb}{0,0.5,0}
\definecolor{pred}{rgb}{0.9,0,0}
\definecolor{pgrey}{rgb}{0.46,0.45,0.48}
\usepackage{inconsolata}

\newcommand{\pArm}{\textit{p}--Arm}

\title{Anteproyecto del \pArm{}}
\author{Javier Alonso Silva \\
        José Alejandro Moya Blanco \\
        Mihai Octavian Stanescu}
\subtitle{Universidad Politécnica de Madrid}
\date{Febrero, 2020}

\begin{document}

    \maketitle
    
    \begin{abstract}
        Se pretende desarrollar un brazo robótico basado en el $\mu$Arm, el cual estará impreso en 3D. El proyecto se plantea desde el punto de vista ingenieril, contemplando al completo el proceso de desarrollo \textit{software} como \textit{hardware}: especificación de alto nivel, desarrollo de cada una de las partes que lo conforman, implementación y pruebas.
        
        Se plantea hacer el sistema impreso en 3D para poder permitir que una persona pueda replicar este proyecto, mejorarlo y ampliarlo en la medida de sus capacidades. Del mismo modo, se pretende que pueda ser accesible a cualquiera que busque iniciarse en el mundo de la robótica, preparando distintos componentes \textit{hardware} y \textit{software} listos para su uso.
    \end{abstract}
    
    \section{Descripción detallada del \textit{hardware}}
    Debido a la naturaleza del proyecto se deberán construir distintos componentes físicos, ya sean mecánicos o electrónicos para conseguir cumplir el objetivo de este. Dichos componentes son la placa de control del brazo robotico y el brazo robotico en si mismo. A continuación se procederá a explicar cada uno de ellos en detalle así como la razón por la que deben ser construidos.

\subsection{Placa de control}

Para conseguir controlar las distintas articulaciones del brazo robotico hará falta diseñar y construir un sistema empotrado que consiga interpretar las órdenes que llegan desde el ordenador de control y pueda generar señales adecuadas para controlar los motores en base a estas ordenes. Este sistema empotrado se concretiza en una placa de control cuyo chip será un DSPIC33EP512GM604. Se ha elegido este microchip debido a la cantidad de canales PWM de los que dispone, la velocidad de calculo matricial y la cantidad de memoria.

Por otro lado, también se han tenido que construir el circuito de regulación de voltaje y el circuito necesario para generar el reloj del microchip a partir de un cristal de cuarzo.

\subsection{Brazo robotico}

El brazo robotico que se pretende construir en este proyecto será impreso en 3D casi en su totalidad empleando la impresora que la universidad pone a nuestra disposición. La razón por la cual se ha decidido construirlo mediante una impresora 3D se debe a que se pretende que el proyecto sea fácilmente reproducible. Por otro lado, también es una opción mas barata en comparación con el aluminio, por ejemplo. Excepción son los ejes de giro de las distintas articulaciones los cuales serán metálicos debido a las cualidades que este material ofrece. 

    
    \section{Recursos necesarios para el desarrollo del proyecto}
    
Dado que este proyecto pretende la construcción de una brazo robótico, los recursos se descompondrán en recursos \textit{hardware}, \textit{software}, materiales de construcción y motores:
\begin{itemize}
    \item Recursos \textit{software}: existen dos componentes \textit{software} principales a desarrollar durante el proyecto, los cuales son la aplicación de control del brazo que se ejecutará en un ordenador auxiliar (Sistema 1) y el \textit{software} de control que se ejecutará en el microcontrolador (Sistema 2) que controla el movimiento del brazo robótico.\\ 
    Los recursos \textit{software} necesarios para el desarrollo de este proyecto son prácticamente inexistentes, ya que pueden ser desarrollados en cualquier ordenador personal, por lo que no hay que añadir nada a requisitos o presupuestos.
    
    \item Recursos \textit{hardware}: el componente principal a desarrollar dentro de esta categoría es la placa de circuito impreso (PCB) que controla el movimiento del brazo robótico. Esta PCB aloja el microcontrolador, el cual procesa las instrucciones recibidas del ordenador auxiliar y controla el movimiento de los motores.\\
    Los recursos \textit{hardware} necesarios son pues el microcontrolador dsPIC33EP512GM604, así como los materiales necesarios para la construcción de la PCB. Ambos recursos están disponibles y pueden ser proporcionados el tutor del proyecto. 
    
    \item Materiales de construcción: la estructura del brazo robótico va a ser impresa íntegramente en 3D, utilizando un material plástico llamado ABS. Cabe destacar que las juntas y ejes móviles del brazo robótico van a ser construidas utilizando rodamientos, pasadores, turcas y tornillos  metálicos.\\
    El principal recurso necesario para la construcción de la estructura es la disponibilidad de una impresora 3D. Actualmente, se dispone de una impresora \textit{Ultimaker 3 Extended} en la escuela, la cual tiene capacidad de imprimir en material ABS y cuenta con volumen de impresión suficiente como para imprimir todas y cada una de las piezas de la estructura. En cuanto a los materiales metálicos necesarios para la construcción de las juntas, se incluyen en el presupuesto y pueden ser adquiridos en cualquier establecimiento especializado.
    
    \item Motores: para llevar a cabo el movimiento del brazo robótico se usarán servomotores ``\textit{Parallax}'' de baja potencia, los cuales pueden ser adquiridos en ``\textit{RS Online}'' y cuentan con disponibilidad inmediata.
\end{itemize}


    
    \section{Presupuesto detallado}
    Para el desarrollo del brazo robótico, serán necesarios los siguientes materiales:

\begin{table}[H]
    \centering
    \begin{tabularx}{\textwidth}{| X | X | c | c | c |}
        \hline
        \textbf{Elemento} & \textbf{Descripción} & \textbf{Cantidad} & \textbf{Precio} & \textbf{Código RS} \\
        \hline
        Servomotor Parallax Inc. & Los servomotores permitirán controlar los distintos movimientos que puede realizar el robot. Será necesario de disponer de varios para poder controlar los tres grados de libertad de los que dispondrá el \pArm{}: uno para la base, otro para el primer segmento y un último para el segundo segmento. & 3 & $13,23$ \EUR{} & 781-3058 \\
        \hline
        Rodamiento de bolas NMB, Radial. & La unión entre ejes del brazo robótico se unen mediante rodamientos, permitiendo así un movimiento fluido y duradero del brazo. & 20 & $4,51$ \EUR{} & 612-6035 \\
        \hline
        Fungibles & Gastos extras que se estipula que puedan haber (por ejemplo, tornillos, tuercas, otros materiales, etc.) pero que no se pueden detallar a estas alturas del proyecto & -- & $50$ \EUR{} & -- \\
        \hline\hline
        \multicolumn{3}{| l |}{\textbf{Total:}} & \multicolumn{2}{ l |}{$179,89$ \EUR{}} \\
        \hline
    \end{tabularx}
    \caption{Tabla completa de presupuestos.}
    \label{tab:budgets}
\end{table}
    
    \section{Plan de trabajo}
    Para conseguir completar este proyecto se ha hecho un plan de trabajo con el que se pretende tener un control sobre el tiempo empleado en cada fase del desarrollo del proyecto, poniendo fechas límite que ayuden a completar los objetivos a tiempo.

\begin{itemize}
    \item \textbf{Del 1 de febrero al 1 de marzo:} se fijan los objetivos del proyecto y se estudia el problema.
    \item \textbf{Del 1 de marzo al 15 de marzo:} se define la estructura del proyecto y de la memoria. Se decide qué se explicará en la memoria y cómo se pretende presentar esta información.
    \item \textbf{Del 15 de marzo al 15 de junio:} se desarrolla el proyecto. En esta fase se empezará el trabajo efectivo del proyecto, se construirá la placa de control y el brazo robótico, se desarrollará el \textit{software} de control y se integrarán los distintos subsistemas. La memoria se desarrollará a la par que la parte práctica del proyecto.
    \item \textbf{Día 15 de junio:} se realiza la primera entrega oficial al tutor de la asignatura.
    \item \textbf{Del 15 de junio a 1 de julio:} periodo de revisión y mejora del proyecto. Se espera retroalimentación del tutor y se mejorará el proyecto en base a dicha retroalimentación.
    \item \textbf{Día 1 de julio:} convocatoria del TFG.
    \item \textbf{Del 1 de julio al 15 de julio:} revisión por parte del tribunal, se realizan los tramites administrativos necesarios y se imprimen los ejemplares necesarios del TFG.
    \item \textbf{Día 15 de julio:} lectura del TFG por parte del equipo de desarrollo ante el tribunal.
    \item \textbf{Día 16 de julio:} final del proyecto.
\end{itemize}

Las fechas y los periodos anteriormente mencionados están sujetos a cambios a lo largo del desarrollo del proyecto para poder adaptarnos a imprevistos que puedan surgir o a limitaciones ajenas al equipo de trabajo.
    
    \section{Configuración del sistema para exposición}
    \input{RecursosYExposicion/exposición}

\end{document}
