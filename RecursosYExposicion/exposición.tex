Dado que durante el proyecto se desarrollará un brazo robótico, se considera que el mismo podría ser expuesto en funcionamiento y realizando una secuencia de movimientos de demostración.

Para ello, se debe disponer de varios elementos, los cuales son: un ordenador auxiliar, el brazo robótico junto con su placa de control y por último una fuente de alimentación. A continuación, se describen dos posibles situaciones de demostración:
\begin{itemize}
    \item Demostración controlada por el usuario, interactuando con el brazo robótico mediante el ordenador auxiliar.
    \item Demostración automática en la cual el usuario no controla los movimientos del brazo robótico.
\end{itemize}

Como posible mejora, se plantea la posibilidad de exponer el brazo robótico sin necesidad de incluir el ordenador auxiliar, ya que se podría programar una secuencia de acciones en el microcontrolador del brazo robótico para  su posterior ejecución. Esta configuración es experimental y aún no se puede confirmar su viabilidad.

Cabe destacar que, durante la demostración, el brazo robótico puede realizar movimientos variados así como desplazar una carga de poco peso.



