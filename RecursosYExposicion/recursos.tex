
Dado que este proyecto pretende la construcción de una brazo robótico, los recursos se descompondrán en recursos \textit{hardware}, \textit{software}, materiales de construcción y motores:
\begin{itemize}
    \item Recursos \textit{software}: existen dos componentes \textit{software} principales a desarrollar durante el proyecto, los cuales son la aplicación de control del brazo que se ejecutará en un ordenador auxiliar (Sistema 1) y el \textit{software} de control que se ejecutará en el microcontrolador (Sistema 2) que controla el movimiento del brazo robótico.\\ 
    Los recursos \textit{software} necesarios para el desarrollo de este proyecto son prácticamente inexistentes, ya que pueden ser desarrollados en cualquier ordenador personal, por lo que no hay que añadir nada a requisitos o presupuestos.
    
    \item Recursos \textit{hardware}: el componente principal a desarrollar dentro de esta categoría es la placa de circuito impreso (PCB) que controla el movimiento del brazo robótico. Esta PCB aloja el microcontrolador, el cual procesa las instrucciones recibidas del ordenador auxiliar y controla el movimiento de los motores.\\
    Los recursos \textit{hardware} necesarios son pues el microcontrolador dsPIC33EP512GM604, así como los materiales necesarios para la construcción de la PCB. Ambos recursos están disponibles y pueden ser proporcionados el tutor del proyecto. 
    
    \item Materiales de construcción: la estructura del brazo robótico va a ser impresa íntegramente en 3D, utilizando un material plástico llamado ABS. Cabe destacar que las juntas y ejes móviles del brazo robótico van a ser construidas utilizando rodamientos, pasadores, turcas y tornillos  metálicos.\\
    El principal recurso necesario para la construcción de la estructura es la disponibilidad de una impresora 3D. Actualmente, se dispone de una impresora \textit{Ultimaker 3 Extended} en la escuela, la cual tiene capacidad de imprimir en material ABS y cuenta con volumen de impresión suficiente como para imprimir todas y cada una de las piezas de la estructura. En cuanto a los materiales metálicos necesarios para la construcción de las juntas, se incluyen en el presupuesto y pueden ser adquiridos en cualquier establecimiento especializado.
    
    \item Motores: para llevar a cabo el movimiento del brazo robótico se usarán servomotores ``\textit{Parallax}'' de baja potencia, los cuales pueden ser adquiridos en ``\textit{RS Online}'' y cuentan con disponibilidad inmediata.
\end{itemize}

